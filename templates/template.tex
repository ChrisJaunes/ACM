\documentclass[10pt]{ctexart}
\usepackage{listings}
\usepackage{xcolor}
\usepackage{xeCJK}
\usepackage{fontspec}
\usepackage{titlesec}
\usepackage{titletoc}
\usepackage{setspace}
\usepackage{graphicx}
\usepackage{geometry}
\geometry{a4paper,scale=0.8}
\renewcommand\contentsname{Contents}
\setmonofont[Mapping={}]{Monaco}    %英文引号之类的正常显示,相当于设置英文字体
%\setsansfont{Consolas} %设置英文字体 Monaco, Consolas,  Fantasque Sans Mono
%\setmainfont{Monaco} %设置英文字体
% 定义可能使用到的颜色
\definecolor{CPPLight}  {HTML} {686868}
\definecolor{CPPSteel}  {HTML} {888888}
\definecolor{CPPDark}   {HTML} {262626}
\definecolor{CPPBlue}   {HTML} {4172A3}
\definecolor{CPPGreen}  {HTML} {487818}
\definecolor{CPPBrown}  {HTML} {A07040}
\definecolor{CPPRed}    {HTML} {AD4D3A}
\definecolor{CPPViolet} {HTML} {7040A0}
\definecolor{CPPGray}  {HTML} {B8B8B8}
\lstset{
    columns=fixed,       
    numbers=left,                                        % 在左侧显示行号
    frame=none,                                          % 不显示背景边框
    backgroundcolor=\color[RGB]{245,245,244},            % 设定背景颜色
    keywordstyle=\color[RGB]{40,40,255},                 % 设定关键字颜色
    numberstyle=\small\color{darkgray},                  % 设定行号格式
    commentstyle=\it\color[RGB]{0,96,96},                % 设置代码注释的格式
    stringstyle=\rmfamily\slshape\color[RGB]{128,0,0},   % 设置字符串格式
    showstringspaces=false,                              % 不显示字符串中的空格
    language=c++,                                        % 设置语言
    morekeywords={alignas,continute,friend,register,true,alignof,decltype,goto,
    reinterpret_cast,try,asm,defult,if,return,typedef,auto,delete,inline,short,
    typeid,bool,do,int,signed,typename,break,double,long,sizeof,union,case,
    dynamic_cast,mutable,static,unsigned,catch,else,namespace,static_assert,using,
    char,enum,new,static_cast,virtual,char16_t,char32_t,explict,noexcept,struct,
    void,export,nullptr,switch,volatile,class,extern,operator,template,wchar_t,
    const,false,private,this,while,constexpr,float,protected,thread_local,
    const_cast,for,public,throw,std},
    emph={map,set,multimap,multiset,unordered_map,unordered_set,
    unordered_multiset,unordered_multimap,vector,string,list,deque,
    array,stack,forwared_list,iostream,memory,shared_ptr,unique_ptr,
    random,bitset,ostream,istream,cout,cin,endl,move,default_random_engine,
    uniform_int_distribution,iterator,algorithm,functional,bing,numeric,},
    emphstyle=\color{CPPViolet},
    basicstyle=\linespread{1}\small\fontspec{Courier New Bold}\ttfamily,
    breaklines=true,
    %xleftmargin=1em,xrightmargin=1em, aboveskip=1em,
}
%\setmainfont{Courier New Bold}
%\begin{lstlisting}

%\end{lstlisting}
\CTEXoptions[today=old]
\title{TEMPLATE}
\author{}
\date{Last build at \today}

\begin{document}{
\begin{titlepage}
    \centering
    {\scshape\Large South China University of Technology \par}
    \vspace{1cm}
    {\scshape\Large SCUT\_gugugu\par}
    \vspace{1.5cm}
    {\Huge\bfseries TEMPLATE\par}
    \vspace{3.5cm}
    \includegraphics[width=0.5\textwidth]{./LOGO}\par\vspace{1cm}
    %\vspace{5cm}
    %{\Large\itshape Nickwzk\par}

    \vfill

% Bottom of the page
    {\large Last build at \today\par}

%\setcounter{page}{0}
\thispagestyle{empty}
%\newpage
\end{titlepage}
\tableofcontents
\newpage
\section{Graph Theory}
\subsection{Shortest Path}
\subsubsection{Dijkstra}
\begin{lstlisting}
typedef pair<int, int> P;
struct Edge {
    int to, nxt;
    LL w;
}e[MAXM];
int head[MAXN], ecnt;
LL d[MAXN];
priority_queue<P, vector<P>, greater<P> > q;
inline void addEdge(int x, int y, LL w) {
    e[++ecnt] = (Edge) {y, head[x], w}; head[x] = ecnt;
}
void dijkstra(int st) {
    memset(d, 0x3f, sizeof(d));
    d[st] = 0;
    q.push(make_pair(0, st));
    while(!q.empty()) {
        P x = q.top(); q.pop();
        int u = x.second;
        for(int i = head[u], v; i; i = e[i].nxt) {
            v = e[i].to;
            if(d[v] > d[u] + e[i].w) {
                d[v] = d[u] + e[i].w;
                q.push(make_pair(d[v], v));
            }
        }
    }
}
\end{lstlisting}
\subsubsection{SPFA}
\begin{lstlisting}
struct Edge {
    int to, nxt;
    LL w;
}e[MAXE];
int head[MAXN], ecnt;
LL d[MAXN];
bool exist[MAXN];
queue<int> q;
inline void addEdge(int x, int y, LL w) {
   e[++ecnt] = (Edge) {y, head[x], w}; head[x] = ecnt;
}
void SPFA(int st) {
    memset(d,0x3f,sizeof(d));
    d[st] = 0;
    q.push(st);
    exist[st] = 1;
    while(!q.empty()) {
        int u = q.front(); q.pop();
        exist[u] = 0;
        for(int i = head[u], v; i; i = e[i].nxt) {
            v = e[i].to;
            if(d[v] > d[u] + e[i].w) {
                d[v] = d[u] + e[i].w;
                //pre[v] = u;
                if(!exist[v]) {
                    q.push(v);
                    exist[v] = 1;
                }
            }
        }
    }
}
\end{lstlisting}
\subsection{Network Flow}
\subsubsection{ISAP}
\begin{lstlisting}
namespace NWF {
    struct Edge{
        int to, nxt;LL f;
    }e[MAXM << 1];
    int S, T, tot;
    int ecnt, head[MAXN], cur[MAXN], pre[MAXN], num[MAXN], dis[MAXN];
    queue<int> q;
    void init(int _S, int _T, int _tot){
        ecnt = 1; S = _S; T = _T; tot = _tot;
        memset(num,  0, (tot + 1) * sizeof(int));
        memset(head, 0, (tot + 1) * sizeof(int));
    } 
    inline void addEdge(int u, int v, LL f) {
        e[++ecnt] = (Edge) {v, head[u], f}; head[u] = ecnt;
        e[++ecnt] = (Edge) {u, head[v], 0}; head[v] = ecnt;
    }
    void bfs() {
        memset(dis, 0, (tot + 1) * sizeof(int));
        q.push(T);
        dis[T] = 1;
        while(!q.empty()) {
            int u = q.front(), v; q.pop();
            num[dis[u]]++;
            for(int i = cur[u] = head[u]; i; i = e[i].nxt) {
                if(!dis[v = e[i].to]) {
                    dis[v] = dis[u] + 1;
                    q.push(v);
                }
            }
        }
    }
    LL augment() {
        LL flow = INF;
        for(int i = S; i != T; i = e[cur[i]].to) 
            flow = min(flow, e[cur[i]].f);
        for(int i = S; i != T; i = e[cur[i]].to) {
            e[cur[i]].f -= flow;
            e[cur[i] ^ 1].f += flow;
        }
        return flow;
    }
    LL isap() {
        bfs();
        int u = S, v;
        LL flow = 0;
        while(dis[S] <= tot) {
            if(u == T) {
                flow += augment();
                u = S;
            }
            bool fg = 0;
            for(int i = cur[u]; i; i = e[i].nxt) {
                if(e[i].f && dis[u] > dis[v = e[i].to]) {
                    pre[v] = u;
                    cur[u] = i;
                    u = v;
                    fg = 1;
                    break;
                }
            }
            if(fg) continue;
            if(!--num[dis[u]]) break;
            int maxDis = tot;
            for(int i = head[u]; i; i = e[i].nxt) {
                if(e[i].f && maxDis > dis[v = e[i].to]) {
                    maxDis = dis[v];
                    cur[u] = i;
                }
            }
            num[dis[u] = maxDis + 1]++;
            if(u != S) u = pre[u];
        }
        return flow;
    }
}
\end{lstlisting}
\subsubsection{HLPP}
\begin{lstlisting}
namespace NWF{
    struct Edge{
        int to,nxt;LL f;
    }e[MAXM << 1];
    int S, T, tot;
    int ecnt, head[MAXN], dis[MAXN], num[MAXN];
    LL sumf[MAXN];
    queue<int> q;
    list<int> dep[MAXN];
    void init(int _S,int _T,int _tot){
        ecnt = 1;S = _S;T = _T;tot = _tot;
        memset(num,  0, (tot + 1) * sizeof(int));
        memset(head, 0, (tot + 1) * sizeof(int));
        memset(sumf, 0, (tot + 1) * sizeof(LL));
    }
    void addEdge(int u,int v,LL f){
        e[++ecnt] = (Edge) {v, head[u], f};head[u] = ecnt;
        e[++ecnt] = (Edge) {u, head[v], 0};head[v] = ecnt;
    }
    void bfs(){
        memset(dis, 0, (tot + 1) * sizeof(int));
        q.push(T); dis[T] = 1;
        while(!q.empty()){
            int u=q.front(), v; q.pop();
            for(int i = head[u]; i; i = e[i].nxt)
            if(!dis[v = e[i].to]){
                dis[v] = dis[u] + 1;
                q.push(v);
            }
        }
    }
    LL hlpp(){
        bfs();
        dis[S] = tot + 1;
        for(int i = 1;i <= tot; ++i)num[dis[i]]++;
        for(int i = tot + 1; ~i; --i)dep[i].clear();
        int maxd = dis[S];LL f;
        dep[maxd].push_back(S);sumf[S] = INF;
        for(;;){
            while(maxd && dep[maxd].empty())maxd--;
            if(!maxd)break;
            int u = dep[maxd].back(), v;dep[maxd].pop_back();
            int minDis = tot + 1;
            for(int i = head[u]; i;i = e[i].nxt)
            if(e[i].f){
                if(dis[u] > dis[v = e[i].to]){
                    f = min(sumf[u], e[i].f);
                    e[i].f -= f;e[i^1].f += f;
                    if(sumf[u] != INF) sumf[u] -= f;
                    if(sumf[v] != INF) sumf[v] += f; 
                    if(v!=S && v!=T && sumf[v] == f){
                        maxd = max(maxd, dis[v]);
                        dep[dis[v]].push_back(v);
                    }
                    if(!sumf[u])break;
                }else minDis=min(minDis, dis[v] + 1);
            }
            if(sumf[u]){
                if(!--num[dis[u]]){
                    for(int i = dis[u];i <= maxd;++i){
                        while(!dep[i].empty()){
                            --num[i];
                            dis[dep[i].back()] = tot + 1;
                            dep[i].pop_back();
                        }
                    }
                    maxd = dis[u] - 1;dis[u] = tot + 1;
                }else{
                    dis[u] = minDis;
                    if(minDis > tot)continue;
                    num[minDis]++;
                    maxd = max(maxd, minDis);
                    dep[minDis].push_back(u);
                }
            }
        }
        return sumf[T];
    }
}
\end{lstlisting}
\subsubsection{Dinic}
\begin{lstlisting}
namespace NWF {
    struct Edge {
        int to, nxt;LL f;
    } e[MAXM << 1]; 
    int S, T, tot;
    int ecnt, head[MAXN], cur[MAXN], dis[MAXN];
    queue<int> q;
    void init(int _S, int _T, int _tot){
        ecnt = 1; S = _S; T = _T; tot = _tot;
        memset(head, 0, (tot + 1) * sizeof(int));
    } 
    void addEdge(int u, int v, LL f) {
        e[++ecnt] = (Edge) {v, head[u], f}; head[u] = ecnt;
        e[++ecnt] = (Edge) {u, head[v], 0}; head[v] = ecnt;
    }
    bool bfs() {
        memset(dis, 0, (tot + 1) * sizeof(int));
        q.push(S); dis[S] = 1;
        while (!q.empty()) {
            int u = q.front(), v; q.pop();
            for (int i = cur[u] = head[u]; i ; i = e[i].nxt) {
                if (e[i].f && !dis[v = e[i].to]) {
                    q.push(v);
                    dis[v] = dis[u] + 1;
                }
            }
        }
        return dis[T];
    }
    LL dfs(int u, LL maxf) {
        if (u == T) return maxf;
        LL sumf = maxf;
        for (int &i = cur[u]; i; i = e[i].nxt) {
            if (e[i].f && dis[e[i].to] > dis[u]) {
                LL tmpf = dfs(e[i].to, min(sumf, e[i].f));
                e[i].f -= tmpf; e[i ^ 1].f += tmpf;
                sumf -= tmpf;
                if (!sumf) return maxf;
            }
        }
        return maxf - sumf;
    }
    LL dinic() {
        LL ret = 0;
        while (bfs()) ret += dfs(S, INF);
        return ret;
    }
}
\end{lstlisting}
\subsubsection{MCMF}
\begin{lstlisting}
namespace NWF{
    struct Edge {
        int to, nxt;LL f, c;
    } e[MAXM << 1]; 
    int S, T, tot;
    int ecnt, head[MAXN], cur[MAXN];LL dis[MAXN];
    bool exist[MAXN];
    queue<int> q;
    void init(int _S, int _T, int _tot){
        ecnt = 1; S = _S; T = _T; tot = _tot;
        memset(head, 0, (tot + 1) * sizeof(int));
    } 
    void addEdge(int u, int v, LL f, LL c) {
        e[++ecnt] = (Edge) {v, head[u], f, c}; head[u] = ecnt;
        e[++ecnt] = (Edge) {u, head[v], 0,-c}; head[v] = ecnt;
    }
    bool spfa() {
        for(int i = 0;i <= tot; ++i){
            dis[i] = INF;exist[i] = cur[i] = 0;
        }
        q.push(S);dis[S] = 0;exist[S] = 1;
        while(!q.empty()) {
            int u = q.front(), v; q.pop();exist[u] = 0;
            for(int i = head[u]; i; i = e[i].nxt) {
                if(e[i].f && dis[v = e[i].to] > dis[u] + e[i].c) {
                    dis[v] = dis[u] + e[i].c;
                    cur[v] = i;
                    if(!exist[v]) {
                        q.push(v);
                        exist[v] = 1;
                    }
                }
            }
        }
        return dis[T] != INF;
    }
    LL mcmf() {
        LL cost = 0;
        while(spfa()) {
            LL flow = INF;
            for(int i = T; i != S; i = e[cur[i] ^ 1].to)
                flow = min(flow, e[cur[i]].f);
            for(int i = T; i != S; i = e[cur[i] ^ 1].to) {
                e[cur[i]].f -= flow;
                e[cur[i] ^ 1].f += flow;
            }
            cost += flow * dis[T];
        }
        return cost;
    }
}
\end{lstlisting}
\subsection{Tree Related}
\subsubsection{Kruskal}
\begin{lstlisting}
namespace MST{
    struct Edge{
        int u,v; LL w;
        bool operator < (const Edge& x) const { return w < x.w; }
    }e[MAXM];
    int ecnt, fa[MAXN];
    void addEdge(int u, int v, LL w) {
        e[++ecnt] = (Edge){v, u, w}; headp[u] = ecnt;
    }
    int Find(int x) { return x == fa[x] ? x : fa[x] = Find(fa[x]); }
    LL kruskal(int n) {
        sort(e + 1, e + ecnt + 1);
        for(int i = 1; i <= n; i++) fa[i] = i;
        LL sum = 0;
        for (int i = 1; i <= ecnt; i++){
            int fu = Find(e[i].u), fv = Find(e[i].v);
            if(fu != fv){
                fa[fu] = fv;
                sum += e[i].w;
            }
        }
        return sum;
    }
}
\end{lstlisting}
\subsubsection{Prim}
\begin{lstlisting}
namespace MST {
    struct Edge{
        int to,nxt; LL w;
    }e[MAXM];
    int ecnt, head[MAXN], vis[MAXN]; // pre[MAXN];
    LL dis[MAXN];
    void addEdge(int u, int v, LL w){
        e[++ecnt] = (Edge){v, head[u], w}; head[u] = ecnt;
        e[++ecnt] = (Edge){u, head[v], w}; head[v] = ecnt;
    }
    LL Prim(int n){
        for (int i = 1; i <= n; i++){
            //pre[i] = 0;
            vis[i] = 0;
            dis[i] = INF;
        }
        vis[1] = 1;
        LL sum = 0;
        for (int i = head[1]; i; i = e[i].nxt)
            dis[e[i].to] = min(dis[e[i].to],e[i].w);
        for (int j = 1; j < n; j++){
            int u; LL minDis = INF;
            for (int i = 1; i <= n; ++i)
                if (!vis[i] && dis[i] < minDis){
                    minDis = dis[i];
                    u = i;
                }
            if (minDis == INF) return -1;
            vis[u] = 1;
            sum += minDis;
            for (int i = head[u], v; i; i = e[i].nxt)
            if (!vis[v = e[i].to] && e[i].w < dis[v]){
                //pre[u] = v;
                dis[v] = e[i].w;
            }
        }
        return sum;
    }
}
\end{lstlisting}
\subsubsection{Tree Divide and Conquer}
\begin{lstlisting}
struct Edge {
    int to, nxt, w;
}e[MAXM];
int head[MAXN], ecnt;
int sz[MAXN];
int d[MAXN], t[5], ans;
bool vis[MAXN];
inline void add_edge(int u, int v, int w) {
    e[++ecnt] = (Edge) {v, h[u], w}; head[u] = ecnt;
    e[++ecnt] = (Edge) {u, h[v], w}; head[v] = ecnt;
}
int getsz(int x, int fa) {
    sz[x] = 1;
    for(int i = h[x]; i; i = e[i].nxt) {
        int y = e[i].to;
        if(vis[y] || y == fa) continue;
        sz[x] += getsz(y, x);
    }
    return sz[x];
}
int getrt(int x) {
    int tot = getsz(x, 0) >> 1;
    while(1) {
        int u = -1;
        for(int i = h[x]; i; i = e[i].nxt) {
            int y = e[i].to;
            if(vis[y] || sz[y] > sz[x]) continue;
            if(u == -1 || sz[y] > sz[u]) u = y;
        }
        if(~u && sz[u] > tot) x = u;
        else break;
    }
    return x;
}
void getdep(int x, int fa) {
    t[d[x]]++;
    for(int i = h[x]; i; i = e[i].nxt) {
        int y = e[i].to;
        if(vis[y] || y == fa) continue;
        d[y] = (d[x] + e[i].w) % 3;
        getdep(y, x);
    }
}
int cal(int x, int v) {
    t[0] = t[1] = t[2] = 0;
    d[x] = v % 3;
    getdep(x, 0);
    return t[0] * t[0] + t[1] * t[2] * 2;
}
void solve(int x) {
    vis[x] = 1;
    ans += cal(x, 0);
    for(int i = h[x]; i; i = e[i].nxt) {
        int y = e[i].to;
        if(vis[y]) continue;
        ans -= cal(y, e[i].w);
        solve(getrt(y));
    }
}
int main() {
    solve(getrt(1));
}
\end{lstlisting}
\subsection{LCA}
\subsubsection{Tree Decomposition LCA}
\begin{lstlisting}
int sz[MAXN], dep[MAXN], top[MAXN], fa[MAXN], son[MAXN], num[MAXN], totw;
struct Edge {
    int to, nxt;
}e[MAXN << 1];
int head[MAXN], ecnt;
inline void add_edge(int x, int y) {
    e[++ecnt] = (Edge) {y, head[x]}; head[x] = ecnt;
}
void dfs1(int x) {
    sz[x] = 1; son[x] = 0;
    for(int i = head[x]; i; i = e[i].nxt) {
        int v = e[i].to;
        if(v == fa[x]) continue;
        fa[v] = x;
        dep[v] = dep[x] + 1;
        dfs1(v);
        sz[x] += sz[v];
        if(sz[v] > sz[son[x]]) son[x] = v;
    }
}
void dfs2(int x) {
    B[num[x]] = A[x];
    if(son[x]) {
        top[son[x]] = top[x];
        num[son[x]] = ++totw;
        dfs2(son[x]);
    }
    for(int i = head[x]; i; i = e[i].nxt) {
        int v = e[i].to;
        if(v == fa[x] || v == son[x]) continue;
        top[v] = v;
        num[v] = ++totw;
        dfs2(v);
    }
}
int lca(int u, int v) {
    if(u == v) return u;
    while(top[u] != top[v]) {
        if(dep[top[u]] > dep[top[v]]) swap(u, v);
        v = fa[top[v]];
    }
    if(dep[u] > dep[v]) swap(u, v);
    return u;
}
inline void init() {
    memset(head, 0, sizeof(head)); ecnt = 0;
    fa[1] = 0; dep[1] = 1; top[1] = 1; num[1] = 1; totw = 1;
}
inline void pre() {
    dfs1(1); dfs2(1);
}
\end{lstlisting}
\subsubsection{Tarjan LCA}
\begin{lstlisting}

\end{lstlisting}
\subsection{Tarjan}
\subsubsection{SCC}
\begin{lstlisting}
namespace SCC{
    vector<int> G[MAXN];
    int dfs_clock, scc_cn, dfn[MAXN], low[MAXN], sccno[MAXN];
    stack<int> S;
    void addEdge(int u, int v) {
        G[u].push_back(v);
    }
    void tarjan(int u) {
        dfn[u] = low[u] = ++dfs_clock;
        S.push(u);
        for(auto v : G[u]) {
            if(!dfn[v]) {
                tarjan(v);
                low[u] = min(low[u], low[v]);
            }else if(!sccno[v]) {
                low[u] = min(low[u], dfn[v]);
            }
        }
        if(dfn[u] == low[u]) {
            scc_cnt++;
            for(;;) {
                int v = S.top(); S.pop();
                sccno[v] = scc_cnt;
                if(v == u) break;
            }
        }
    }
    void findSCC(int n) {
        for(int i = 1; i <= n; i++)
            if(!dfn[i]) tarjan(i);
    }
    void init(int n){
        dfs_clock = scc_cnt = 0;
        for(int i = 0;i <= n;++i){
            dfn[i] = low[i] = sccno[i] = 0;
            G[i].clear();
        }
    }
}
\end{lstlisting}
\subsubsection{BCC}
\begin{lstlisting}
struct Edge {
    int to, nxt;
}e[MAXE];
struct Node {
    int u, v;
};
int head[MAXN], ecnt;
int pre[MAXN], low[MAXN], iscut[MAXN], bccno[MAXN], dfs_clock, bcc_cnt;
vector<int> bcc[MAXN];
stack<Node> S;
inline void add_edge(int x, int y) {
    e[++ecnt] = (Edge) {y, head[x]}; head[x] = ecnt;
    e[++ecnt] = (Edge) {x, head[y]}; head[y] = ecnt;
}
inline void init() {
    memset(pre, 0, sizeof(pre));
    memset(low, 0, sizeof(low));
    memset(bccno, 0, sizeof(bccno));
    memset(iscut, 0, sizeof(iscut));
    memset(head, 0, sizeof(head)); ecnt = 0;
    dfs_clock = bcc_cnt = 0;
}
void tarjan(int u, int fa) {
    low[u] = pre[u] = ++dfs_clock;
    int ch = 0;
    for(int i = head[u]; i; i = e[i].nxt) {
        int v = e[i].to;
        if(!pre[v]) {
            S.push((Node) {u, v});
            ch++;
            tarjan(v, u);
            low[u] = min(low[u], low[v]);
            if(low[v] >= pre[u]) {
                iscut[u] = 1;
                bcc[bcc_cnt++].clear();
                for(;;) {
                    Node x = S.top(); S.pop();
                    if(bccno[x.u] != bcc_cnt) {
                        bcc[bcc_cnt].push_back(x.u);
                        bccno[x.u] = bcc_cnt;
                    }
                    if(bccno[x.v] != bcc_cnt) {
                        bcc[bcc_cnt].push_back(x.v);
                        bccno[x.v] = bcc_cnt;
                    }
                    if(x.u == u && x.v == v) break;
                }
            }
        }
        else if(pre[v] < pre[u] && v != fa) {
            S.push((Node) {u, v});
            low[u] = min(low[u], pre[v]);
        }
    }
    if(u == fa && ch <= 1) iscut[u] = 0;
}
\end{lstlisting}





\newpage
\section{Data Structures}
\subsection{Basic Structures}
\begin{lstlisting}

\end{lstlisting}
\subsection{Tree Structures}
\subsubsection{Tree Decomposition}
\begin{lstlisting}
int sz[MAXN], dep[MAXN], top[MAXN], fa[MAXN], son[MAXN], num[MAXN], totw;
struct Edge {
    int to, nxt;
}e[MAXN << 1];
int head[MAXN], ecnt;
int n, m, Q;
#define Ls(x) (x << 1)
#define Rs(x) (x << 1 | 1)
struct Tree {
    int l, r, lazy;
    LL sum, mx;
}tree[MAXN << 2];
int A[MAXN], B[MAXN];
void push_up(int x) {
    tree[x].sum = tree[Ls(x)].sum + tree[Rs(x)].sum;
    tree[x].mx = max(tree[Ls(x)].mx, tree[Rs(x)].mx);
}
void push_down(int x) {
    if(tree[x].lazy) {
        tree[Ls(x)].sum += tree[x].lazy * (tree[Ls(x)].r - tree[Ls(x)].l + 1);
        tree[Rs(x)].sum += tree[x].lazy * (tree[Rs(x)].r - tree[Rs(x)].l + 1);
        tree[Ls(x)].mx += tree[x].lazy;
        tree[Rs(x)].mx += tree[x].lazy;
        tree[Ls(x)].lazy += tree[x].lazy;
        tree[Rs(x)].lazy += tree[x].lazy;
        tree[x].lazy = 0;
    }
}
void build(int x, int L, int R) {
    tree[x].lazy = 0;
    tree[x].l = L; tree[x].r = R;
    if(L == R) {
        tree[x].sum = B[L];
        tree[x].mx = B[L];
        return;
    }
    int mid = (L + R) >> 1;
    build(Ls(x), L, mid);
    build(Rs(x), mid + 1, R);
    push_up(x);
}
void update(int x, int L, int R, LL val) {
    if(tree[x].l >= L && tree[x].r <= R) {
        tree[x].lazy += val;
        tree[x].sum += val * (tree[x].r - tree[x].l + 1);
        tree[x].mx += val;
        return;
    }
    push_down(x);
    int mid = (tree[x].l + tree[x].r) >> 1;
    if(L <= mid) update(Ls(x), L, R, val);
    if(R > mid) update(Rs(x), L, R, val);
    push_up(x);
}
LL query(int x, int L, int R) {
    if(tree[x].l >= L && tree[x].r <= R)
        return tree[x].sum;
    push_down(x);
    int mid = (tree[x].l + tree[x].r) >> 1;
    LL res = 0;
    if(L <= mid) res += query(Ls(x), L, R);
    if(R > mid) res += query(Rs(x), L, R);
    return res;
}
LL query2(int x, int L, int R) {
    if(tree[x].l >= L && tree[x].r <= R)
        return tree[x].mx;
    push_down(x);
    int mid = (tree[x].l + tree[x].r) >> 1;
    LL res = -INF;
    if(L <= mid) res = max(res, query2(Ls(x), L, R));
    if(R > mid) res = max(res, query2(Rs(x), L, R));
    return res;
}
inline void add_edge(int x, int y) {
    e[++ecnt] = (Edge) {y, head[x]}; head[x] = ecnt;
}
void dfs1(int x) {
    sz[x] = 1; son[x] = 0;
    for(int i = head[x]; i; i = e[i].nxt) {
        int v = e[i].to;
        if(v == fa[x]) continue;
        fa[v] = x;
        dep[v] = dep[x] + 1;
        dfs1(v);
        sz[x] += sz[v];
        if(sz[v] > sz[son[x]]) son[x] = v;
    }
}
void dfs2(int x) {
    B[num[x]] = A[x];
    if(son[x]) {
        top[son[x]] = top[x];
        num[son[x]] = ++totw;
        dfs2(son[x]);
    }
    for(int i = head[x]; i; i = e[i].nxt) {
        int v = e[i].to;
        if(v == fa[x] || v == son[x]) continue;
        top[v] = v;
        num[v] = ++totw;
        dfs2(v);
    }
}
void up(int a, int b, int c) {
    int f1 = top[a], f2 = top[b];
    while(f1 != f2) {
        if(dep[f1] < dep[f2]) { swap(a, b); swap(f1, f2); }
        update(1, num[f1], num[a], c);
        a = fa[f1];
        f1 = top[a];
    }
    if(dep[a] > dep[b]) swap(a, b);
    update(1, num[a], num[b], c);
}
int qsum(int a, int b) {
    if(a == b) return query(1, num[a], num[a]);
    int f1 = top[a], f2 = top[b];
    int res = 0;
    while(f1 != f2) {
        if(dep[f1] < dep[f2]) { swap(a, b); swap(f1, f2); }
        res += query(1, num[f1], num[a]);
        a = fa[f1];
        f1 = top[a];
    }
    if(dep[a] > dep[b]) swap(a, b);
    res += query(1, num[a], num[b]);
    return res;
}
int qmax(int a, int b) {
    if(a == b) return query2(1, num[a], num[a]);
    int f1 = top[a], f2 = top[b];
    int res = -1000000000;
    while(f1 != f2) {
        if(dep[f1] < dep[f2]) { swap(a, b); swap(f1, f2); }
        res = max(res, query2(1, num[f1], num[a]));
        a = fa[f1];
        f1 = top[a];
    }
    if(dep[a] > dep[b]) swap(a, b);
    res = max(res, query2(1, num[a], num[b]));
    return res;
}
inline void init() {
    memset(head, 0, sizeof(head)); ecnt = 0;
    fa[1] = 0; dep[1] = 1; top[1] = 1; num[1] = 1; totw = 1;
}
inline void pre() {
    dfs1(1); dfs2(1); build(1, 1, totw);
}
\end{lstlisting}
\subsubsection{Link-Cut Tree}
\begin{lstlisting}
namespace LCT {
    int fa[MAXN], rev[MAXN], tr[MAXN][2];
    int s[MAXN], val[MAXN];
    void push_up(int x) {
        int l = tr[x][0], r = tr[x][1];
        s[x] = s[l] + s[r] + val[x];
    }
    void Rev(int x) {
        rev[x] ^= 1; swap(tr[x][0], tr[x][1]);
    }
    void push_down(int x) {
        if(!rev[x]) return;
        int l = tr[x][0], r = tr[x][1];
        rev[x] = 0;
        if(l) Rev(l); if(r) Rev(r);
    }
    bool isroot(int x) {
        return tr[fa[x]][0] != x && tr[fa[x]][1] != x;
    }
    void pre(int x) {
        if(!isroot(x)) pre(fa[x]);
        push_down(x);
    }
    void rotate(int x) {
        int y = fa[x]; int z = fa[y];
        int l = tr[y][1] == x;
        int r = l ^ 1;
        if(!isroot(y)) tr[z][tr[z][1] == y] = x;
        fa[x] = z; fa[y] = x; fa[tr[x][r]] = y;
        tr[y][l] = tr[x][r]; tr[x][r] = y;
        push_up(y);
    }
    void splay(int x) {
        pre(x);
        int y, z;
        while(!isroot(x)) {
            y = fa[x]; z = fa[y];
            if(!isroot(y)) {
                if((tr[z][0] == y) == (tr[y][0] == x))rotate(y);
                else rotate(x);
            }
            rotate(x);
        }
        push_up(x);
    }
    void access(int x) {
        int y = 0;
        while(x) {
            splay(x); tr[x][1] = y;
            push_up(x);
            y = x; x = fa[x];
        }
    }
    void makeroot(int x) {
        access(x); splay(x); Rev(x);
    }
    void lnk(int x, int y) {
        makeroot(x); fa[x] = y;
    }
    void cut(int x, int y) {
        makeroot(x); access(y); splay(y);
        tr[y][0] = fa[x] = 0; push_up(y);
    }
    void update(int x, int y) {
        makeroot(x); val[x] = y; push_up(x);
    }
    int query(int x, int y) {
        makeroot(x); access(y); splay(y);
        return s[y];
    }
    bool check(int x, int y) {
        int tmp = y;
        makeroot(x); access(y); splay(x);
        while(!isroot(y)) y = fa[y];
        splay(tmp);
        return x == y;
    }
}
\end{lstlisting}
\subsection{Sequence Structures}
\subsubsection{Segment Tree}
\begin{lstlisting}
#define Ls(x) (x << 1)
#define Rs(x) (x << 1 | 1)
struct Tree {
    int l, r, lazy;
    LL sum, mx;
}tree[MAXN << 2];
int A[MAXN];
void push_up(int x) {
    tree[x].sum = tree[Ls(x)].sum + tree[Rs(x)].sum;
    tree[x].mx = max(tree[Ls(x)].mx, tree[Rs(x)].mx);
}
void push_down(int x) {
    if(tree[x].lazy) {
        tree[Ls(x)].sum += tree[x].lazy * (tree[Ls(x)].r - tree[Ls(x)].l + 1);
        tree[Rs(x)].sum += tree[x].lazy * (tree[Rs(x)].r - tree[Rs(x)].l + 1);
        tree[Ls(x)].mx += tree[x].lazy;
        tree[Rs(x)].mx += tree[x].lazy;
        tree[Ls(x)].lazy += tree[x].lazy;
        tree[Rs(x)].lazy += tree[x].lazy;
        tree[x].lazy = 0;
    }
}
void build(int x, int L, int R) {
    tree[x].lazy = 0;
    tree[x].l = L; tree[x].r = R;
    if(L == R) {
        tree[x].sum = A[L];
        tree[x].mx = A[L];
        return;
    }
    int mid = (L + R) >> 1;
    build(Ls(x), L, mid);
    build(Rs(x), mid + 1, R);
    push_up(x);
}
void update(int x, int L, int R, LL val) {
    if(tree[x].l >= L && tree[x].r <= R) {
        tree[x].lazy += val;
        tree[x].sum += val * (tree[x].r - tree[x].l + 1);
        tree[x].mx += val;
        return;
    }
    push_down(x);
    int mid = (tree[x].l + tree[x].r) >> 1;
    if(L <= mid) update(Ls(x), L, R, val);
    if(R > mid) update(Rs(x), L, R, val);
    push_up(x);
}
LL query(int x, int L, int R) {
    if(tree[x].l >= L && tree[x].r <= R)
        return tree[x].sum;
    push_down(x);
    int mid = (tree[x].l + tree[x].r) >> 1;
    LL res = 0;
    if(L <= mid) res += query(Ls(x), L, R);
    if(R > mid) res += query(Rs(x), L, R);
    return res;
}
LL query2(int x, int L, int R) {
    if(tree[x].l >= L && tree[x].r <= R)
        return tree[x].mx;
    push_down(x);
    int mid = (tree[x].l + tree[x].r) >> 1;
    LL res = -INF;
    if(L <= mid) res = max(res, query2(Ls(x), L, R));
    if(R > mid) res = max(res, query2(Rs(x), L, R));
    return res;
}
\end{lstlisting}
\subsubsection{Splay Tree}
\begin{lstlisting}

\end{lstlisting}
\subsection{Persistent Data Structures}
\subsubsection{Chairman Tree}
\begin{lstlisting}
struct Node {
    int l, r;
    LL sum;
}t[MAXN * 40];
int cnt, n;
int rt[MAXN];
void update(int pre, int &x, int l, int r, int v) {
    x = ++cnt; t[x] = t[pre]; t[x].sum++;
    if(l == r) return;
    int mid = (l + r) >> 1;
    if(v <= mid) update(t[pre].l, t[x].l, l, mid, v);
    else update(t[pre].r, t[x].r, mid + 1, r, v);
}
int query(int x, int y, int l, int r, int v) {
    if(l == r) return l;
    int mid = (l + r) >> 1;
    int sum = t[t[y].l].sum - t[t[x].l].sum;
    if(sum >= v) return query(t[x].l, t[y].l, l, mid, v);
    else return query(t[x].r, t[y].r, mid + 1, r, v - sum);
}
\end{lstlisting}





\newpage
\section{String}
\subsection{Basics}
\subsubsection{Hash}
\begin{lstlisting}
const LL p1 = 201, p2 = 301, mod1 = 1200000319, mod2 = 2147483647;
struct Hash {
    LL a, b;
    void append(Hash pre, int v) {
        a = (pre.a * p1 + v) % mod1;
        b = (pre.b * p2 + v) % mod2;
    }
    void init(string S) {
        a = b = 0;
        for(int i = 0; i < S.size(); i++) append(*this, S[i]);
    }
    bool operator == (const Hash &x) const {
        return a == x.a && b == x.b;
    }
    bool operator < (const Hash &x) const {
        return a < x.a || (a == x.a && b < x.b);
    }
};
\end{lstlisting}
\subsubsection{KMP \&\& exKMP}
\begin{lstlisting}
namespace KMP {
    int f[MAXN];
    void get_fail(string A) {
        f[0] = 0; f[1] = 0;
        for(int i = 1; i < A.size(); i++) {
            int j = f[i];
            while(j && A[i] != A[j]) j = f[j];
            f[i + 1] = A[i] == A[j] ? j + 1 : 0;
        }
    }

    void kmp(string A, string B) {
        get_fail(B);
        int j = 0;
        for(int i = 0; i < A.size(); i++) {
            while(j && B[j] != A[i]) j = f[j];
            if(B[j] == A[i]) j++;
            if(j == B.size()) {
                ans++;
                j = f[j];
            }
        }
    }
}
namespace exKMP {
    int nxt[MAXN], ext[MAXN];
    void get_nxt(string T) {
        int j = 0, mx = 0;
        int m = T.size();
        nxt[0] = m;
        for(int i = 1; i < m; i++) {
            if(i >= mx || i + nxt[i - j] >= mx) {
                if(i >= mx) mx = i;
                while(mx < m && T[mx] == T[mx - i]) mx++;
                nxt[i] = mx - i;
                j = i;
            }
            else nxt[i] = nxt[i - j];
        }
    }
    void exkmp(string S, string T) {
        int j = 0, mx = 0;
        get_nxt(T);
        int n = S.size(), m = T.size();
        for(int i = 0; i < n; i++) {
            if(i >= mx || i + nxt[i - j] >= mx) {
                if(i >= mx) mx = i;
                while(mx < n && mx - i < m && S[mx] == T[mx - i]) mx++;
                ext[i] = mx - i;
                j = i;
            }
            else ext[i] = nxt[i - j];
        }
    }
}
\end{lstlisting}
\subsubsection{AC Automaton}
\begin{lstlisting}
namespace AC {
    int ch[MAXN][sigma_size], last[MAXN];
    int val[MAXN], f[MAXN], sz;
    inline void init() { sz = 1; memset(ch[0], 0, sizeof(ch[0])); }
    inline int idx(char c) { return c - 'a'; }
    void insert(string s, int v) {
        int u = 0;
        for(int i = 0; i < s.size(); i++) {
            int c = idx(s[i]);
            if(!ch[u][c]) {
                memset(ch[sz], 0, sizeof(ch[sz]));
                val[sz] = 0;
                ch[u][c] = sz++;
            }
            u = ch[u][c];
        }
        val[u] = v;
    }
    void get_fail() {
        queue<int> q;
        f[0] = 0;
        for(int c = 0; c < sigma_size; c++) {
            int u = ch[0][c];
            if(u) { f[u] = 0; q.push(u); last[u] = 0; }
        }
        while(!q.empty()) {
            int r = q.front(); q.pop();
            for(int c = 0; c < sigma_size; c++) {
                int u = ch[r][c];
                if(!u) { ch[r][c] = ch[f[r]][c]; continue; }
                q.push(u);
                int v = f[r];
                while(v && !ch[v][c]) v = f[v];
                f[u] = ch[v][c];
                last[u] = val[f[u]] ? f[u] : last[f[u]];
            }
        }
    }
    inline void solve(int j) {
        if(j) {
            ans += val[j];
            solve(last[j]);
        }
    }
    void find(string T) {
        int j = 0;
        for(int i = 0; i < T.size(); i++) {
            int c = idx(T[i]);
            j = ch[j][c];
            if(val[j]) solve(j);
            else if(last[j]) solve(last[j]);
        }
    }
}
\end{lstlisting}
\subsection{Suffix Related}
\subsubsection{Suffix Array}
\begin{lstlisting}
namespace SA {
    char s[MAXN];
    int sa[MAXN], rank[MAXN], height[MAXN];
    int t[MAXN], t2[MAXN], c[MAXN], n;
    void clear() { n = 0; memset(sa, 0, sizeof(sa)); }
    void build(int m) {
        int *x = t, *y = t2;
        for(int i = 0; i < m; i++) c[i] = 0;
        for(int i = 0; i < n; i++) c[x[i] = s[i]]++;
        for(int i = 1; i < m; i++) c[i] += c[i - 1];
        for(int i = n - 1; i >= 0; i--) sa[--c[x[i]]] = i;
        for(int k = 1; k <= n; k <<= 1) {
            int p = 0;
            for(int i = n - k; i < n; i++) y[p++] = i;
            for(int i = 0; i < n; i++) if(sa[i] >= k) y[p++] = sa[i] - k;
            for(int i = 0; i < m; i++) c[i] = 0;
            for(int i = 0; i < n; i++) c[x[y[i]]]++;
            for(int i = 1; i < m; i++) c[i] += c[i - 1];
            for(int i = n - 1; i >= 0; i--) sa[--c[x[y[i]]]] = y[i];
            swap(x, y);
            p = 1; x[sa[0]] = 0;
            for(int i = 1; i < n; i++)
                x[sa[i]] = y[sa[i - 1]] == y[sa[i]] && y[sa[i - 1] + k] == y[sa[i] + k] ? p - 1 : p++;
            if(p >= n) break;
            m = p;
        }
    }
    void buildHeight() {
        int k = 0;
        for(int i = 0; i < n; i++) rank[sa[i]] = i;
        for(int i = 0; i < n; i++) {
            if(k) k--;
            int j = sa[rank[i] - 1];
            while(s[i + k] == s[j + k]) k++;
            height[rank[i]] = k;
        }
    }
    void init() {
        n = strlen(s) + 1;
        build('z' + 1);
        buildHeight();
    }
}
\end{lstlisting}
\subsubsection{Suffix Automaton}
\begin{lstlisting}

\end{lstlisting}
\subsection{Palindrome Related}
\subsubsection{Manacher}
\begin{lstlisting}
namespace Palindrome {
    char s1[MAXN], s2[MAXN];
    int len1, len2, ans;
    int p[MAXN]; //p[i] - 1
    void init() {
        len1 = strlen(s1);
        s2[0] = '$';
        s2[1] = '#';
        for(int i = 0; i < len1; i++) {
            s2[2 * i + 2] = s1[i];
            s2[2 * i + 3] = '#';
        }
        len2 = len1 * 2 + 2;
        s2[len2] = '&';
    }
    void manacher() {
        int id = 0, mx = 0;
        for(int i = 1; i < len2; i++) {
            if(mx > i) p[i] = min(p[2 * id - i], mx - i);
            else p[i] = 1;
            while(s2[i + p[i]] == s2[i - p[i]]) p[i]++;
            if(i + p[i] > mx) {
                mx = i + p[i];
                id = i;
            }
        }
    }
}
\end{lstlisting}
\subsubsection{Palindromic Tree}
\begin{lstlisting}

\end{lstlisting}





\newpage
\section{Math}
\subsection{Algebra}
\subsubsection{FFT}
\begin{lstlisting}
const double pi = acos(-1.0);
const int MAXN = 300003;
struct comp {
    double x, y;
    comp operator + (const comp a) const { return (comp) {x + a.x, y + a.y}; }
    comp operator - (const comp a) const { return (comp) {x - a.x, y - a.y}; }
    comp operator * (const comp a) const { return (comp) {x * a.x - y * a.y, x * a.y + y * a.x}; }
};
int rev[MAXN], T;
comp tmp;
void fft(comp *a, int r) {
    if(r == -1) for(int i = 0; i < T; i++) A[i] = A[i] * A[i];
    for(int i = 0; i < T; i++) if(rev[i] > i) swap(a[rev[i]], a[i]);
    for(int i = 2, mid = 1; i <= T; mid = i, i <<= 1) {
        comp step = (comp) {cos(pi / mid), r * sin(pi / mid)};
        for(int j = 0; j < T; j += i) {
            comp cur = (comp) {1, 0};
            for(int k = j; k < j + mid; k++, cur = cur * step) {
                tmp = a[k + mid] * cur;
                a[k + mid] = a[k] - tmp;
                a[k] = a[k] + tmp;
            }
        }
    }
    if(r == -1) for(int i = 0; i < T; i++) a[i].y = (int)(a[i].y / T / 2 + 0.5);
}
int n, m;
comp A[MAXN];
void init() {
    for(T = 1; T <= n + m; T <<= 1);
    for(int i = 1; i < T; i++) {
        if(i & 1) rev[i] = (rev[i >> 1] >> 1) ^ (T >> 1);
        else rev[i] = rev[i >> 1] >> 1;
    }
}
\end{lstlisting}
\subsubsection{NTT}
\begin{lstlisting}
const int MAXN = 300005, G = 3, mod = 998244353; //or (479LL<<21) + 1
int rev[MAXN], T;
LL qpow(LL x, LL y) {
    LL res = 1;
    while(y) {
        if(y & 1) res = res * x % mod;
        x = x * x % mod;
        y >>= 1;
    }
    return res;
}
void ntt(LL *a, int r) {
    if(r == -1) for(int i = 0; i < T; i++) A[i] = A[i] * B[i] % mod;
    for(int i = 0; i < T; i++) if(rev[i] > i) swap(a[rev[i]], a[i]);
    for(int i = 2, mid = 1; i <= T; mid = i, i <<= 1) {
        LL gn = qpow(G, (mod - 1) / i);
        if(r == -1) gn = qpow(gn, mod - 2);
        for(int j = 0; j < T; j += i) {
            LL cur = 1, tmp;
            for(int k = j; k < j + mid; k++, cur = cur * gn % mod) {
                tmp = a[k + mid] * cur % mod;
                a[k + mid] = ((a[k] - tmp) % mod + mod) % mod;
                a[k] = (a[k] + tmp) % mod;
            }
        }
    }
    if(r == -1) {
        LL inv = qpow(T, mod - 2);
        for(int i = 0; i < T; i++) a[i] = a[i] * inv % mod;
    }
}
int n, m;
LL A[MAXN], B[MAXN];
void init() {
    for(T = 1; T <= n + m; T <<= 1);
    for(int i = 0; i < T; i++) {
        if(i & 1) rev[i] = (rev[i >> 1] >> 1) ^ (T >> 1);
        else rev[i] = rev[i >> 1] >> 1;
    }
}
\end{lstlisting}
\subsection{Math Theory}
\subsubsection{CRT \&\& exCRT}
\begin{lstlisting}
namespace CRT {
    LL m[MAXN], a[MAXN]; //x_i = a[i] (mod m[i])
    LL exgcd(LL _a, LL _b, LL &x, LL &y) {
        if(!_b) {
            x = 1; y = 0;
            return _a;
        }
        LL d = exgcd(_b, _a % _b, y, x);
        y -= (_a / _b) * x;
        return d;
    }
    LL crt(int n) {
        LL M = 1, tmp, res = 0, x, y;
        for(int i = 1; i <= n; i++) M *= m[i];
        for(int i = 1; i <= n; i++) {
            tmp = M / m[i];
            exgcd(tmp, m[i], x, y);
            x = (x + m[i]) % m[i];
            res = (a[i] * x % M * tmp % M + res) % M;
        }
        return res;
    }
}
namespace EXCRT {
    LL m[MAXN], a[MAXN];
    LL exgcd(LL _a, LL _b, LL &x, LL &y) {
        if(!_b) {
            x = 1; y = 0;
            return _a;
        }
        LL d = exgcd(_b, _a % _b, y, x);
        y -= (_a / _b) * x;
        return d;
    }
    LL excrt(int n) {
        LL M = m[1], A = a[1], x, y, d, tmp;
        for(int i = 2; i <= n; i++) {
            d = exgcd(M, m[i], x, y);
            if((A - a[i]) % d) return -1; //No solution
            tmp = M / d; M *= m[i] / d;
            y = (A - a[i]) / d % M * y % M;
            y = (y + tmp) % tmp;
            A = (m[i] % M * y % M + a[i]) % M;
            A = (A + M) % M;
        }
        return A;
    }
}
\end{lstlisting}





\newpage
\section{ComputationalGeometry}
\begin{lstlisting}
namespace CG{
    #define Point Vector
    const double pi=acos(-1.0);
    const double inf=1e100;
    const double eps=1e-9;
    template <typename T> inline T Abs(T x){return x>0?x:-x;}
    template <typename T> inline bool operator == (T x,T y){return Abs(x-y)<eps;}
    int sgn(double x){
        if (Abs(x)<eps) return 0;
        if (x>0) return 1;
        else return -1;
    }
}
namespace CG{
    struct Point{
        double x,y;
        Point(double x=0,double y=0):x(x),y(y){}
    };
    Vector operator + (const Vector a,const Vector b){return Vector(a.x+b.x,a.y+b.y);}
    Vector operator - (const Vector a,const Vector b){return Vector(a.x-b.x,a.y-b.y);}
    Vector operator * (const Vector a,const double k){return Vector(a.x*k,a.y*k);}
    Vector operator / (const Vector a,const double k){return Vector(a.x/k,a.y/k);}
    bool operator < (const Vector a,const Vector b) {return a.x==b.x?a.y<b.y:a.x<b.x;}
    bool operator == (const Vector a,const Vector b) {return a.x==b.x && a.y==b.y;}
    double Dot(const Vector a,const Vector b){return a.x*b.x+a.y*b.y;}
    double Cross(const Vector a,const Vector b){return a.x*b.y-a.y*b.x;}    
    double Norm(const Vector a){return sqrt(Dot(a,a));}
    double Angle(const Vector a,const Vector b){return acos(Dot(a,b)/Norm(a)/Norm(b));}
    Vector Rotate(const Vector a,const double theta){return Vector(a.x*cos(theta)-a.y*sin(theta),a.x*sin(theta)+a.y*cos(theta));}
    bool ToLeftTest(const Vector a,const Vector b){return Cross(a,b)<0;}
    double DisPP(const Vector a,const Vector b){return sqrt((a.x-b.x)*(a.x-b.x)+(a.y-b.y)*(a.y-b.y));}
}
namespace CG{
    struct Line{
        Point p0,v,p1;
        double t,theta;
        Line(Point _p0=0,Point _v=0,double _t=1):p0(_p0),v(_v),t(_t){p1=p0+v*t; theta=atan2(v.y,v.x);}
        // Line(Point _p0=0,Point _v=0,double _t=1):p0(_p0),p1(_v){v=(p1-p0)/t; theta=atan2(v.y,v.x);}
    };
    bool operator < (const Line n,const Line m) {return n.theta<m.theta;}
    Point GetIntersection(const Line n,const Line m){return n.p0+n.v*Cross(m.v,(n.p0-m.p0))/Cross(n.v,m.v);}
    bool OnLine(const Vector a,const Line l){return Cross(l.p0-a,l.p1-a)==0;}
    bool OnSegment(const Point a,const Line l){return sgn(Cross(l.p0-a,l.p1-a))==0 && sgn(Dot(l.p0-a,l.p1-a))<0;}
    double DisPL(const Point a,const Line l){return Abs(Cross(l.p1-l.p0,a-l.p0)/Norm(l.p1-l.p0));}
    double DisPS(const Point a,const Line l){
        if (l.p0==l.p1) return Norm(a-l.p0);
        Vector v1=l.p1-l.p0,v2=a-l.p0,v3=a-l.p1;
        if (sgn(Dot(v1,v2))<0) return Norm(v2);
        if (sgn(Dot(v1,v3))>0) return Norm(v3);
        return DisPL(a,l);
    }
    Point GetProjection(const Point a,const Line l){
        Vector v=l.p1-l.p0;
        return l.p0+v*(Dot(v,a-l.p0)/Dot(v,v));
    }
    bool SegmentIntersection(const Line n,const Line m,bool p){
        double c1=Cross(n.p1-n.p0,m.p1-m.p0);
        double c2=Cross(n.p1-n.p0,m.p1-n.p0);
        double c3=Cross(m.p1-m.p0,n.p0-m.p0);
        double c4=Cross(m.p1-m.p0,n.p1-m.p0);
        if (p){
            if (!sgn(c1) || !sgn(c2) || !sgn(c3) || !sgn(c4)){
                return OnSegment(n.p0,m) || OnSegment(n.p1,m) || OnSegment(m.p0,n) || OnSegment(m.p0,m);

            }
        }
        return (sgn(c1)*sgn(c2)<0 && sgn(c3)*sgn(c4)<0);
    }
}
namespace CG{
    Point p[1000010],s[1000010];
    int ConvexHull(Point *p,int n){
        sort(p+1,p+1+n);
        int m=0;
        for (int i=1;i<=n;i++){
            for (;m>=2 && !ToLeftTest(s[m]-s[m-1],p[i]-s[m-1]);m--);
            s[++m]=p[i];
        }
        int k=m;
        for (int i=n-1;i;i--){
            for (;m>=k+1 && !ToLeftTest(s[m]-s[m-1],p[i]-s[m-1]);m--);
            s[++m]=p[i];
        }
        return m;
    }
    double GetArea(Point *p,int n){
        double area=0;
        for (int i=2;i<=n;i++) area+=0.5*Cross(p[i-1],p[i]);
        return Abs(area);
    }
    double GetCircumference(Point *p,int n){
        double Circumference=0;
        for (int i=2;i<=n;i++) Circumference+=DisPP(p[i-1],p[i]);
        return Circumference;
    }
}
namespace CG{
    void HalfPlane(Line l[],int n){
        deque <Point> p;
        sort(l+1,l+1+n);
        deque <Line> q;
        q.push_back(l[1]);
        for (int i=2;i<=n;i++){
            for (;!p.empty() && !ToLeftTest(p.back()-l[i].p0,l[i].v);q.pop_back(),p.pop_back());
            for (;!p.empty() && !ToLeftTest(p.front()-l[i].p0,l[i].v);q.pop_front(),p.pop_front());
            if (sgn(Cross(l[i].v,q.back().v))==0)
                if (ToLeftTest(l[i].p0-q.back().p0),q.back().v){
                    q.pop_back();
                    if (!p.empty()) p.pop_back();
                }
            if (!q.empty()) p.push_back(GetIntersection(q.back(),l[i]));
            q.push_back(l[i]);
        }
        for (;!p.empty() && !ToLeftTest(p.back()-q.front().p0,q.front().v);q.pop_back(),p.pop_back());
        p.push_back(GetIntersection(q.back(),q.front()));
        double area=0.5*Cross(p.back(),p.front()); Point last=p.front();
        for (p.pop_front();!p.empty();last=p.front(),p.pop_front()) area+=0.5*Cross(last,p.front());
        printf("%.1f",Abs(area));
    }
}
namespace CG{
    Point GetCircleCenter(const Point a,const Point b,const Point c){
        Point p=(a+b)/2.0,q=(a+c)/2.0;
        Vector v=Rotate(b-a,pi/2.0),w=Rotate(c-a,pi/2.0);
        if (sgn(Norm(Cross(v,w)))==0){
            if (sgn(Norm(a-b)+Norm(b-c)-Norm(a-c))==0) return (a+c)/2;
            if (sgn(Norm(b-a)+Norm(a-c)-Norm(b-c))==0) return (b+c)/2;
            if (sgn(Norm(a-c)+Norm(c-b)-Norm(a-b))==0) return (a+c)/2;
        }
        return GetIntersection(Line(p,v),Line(q,w));
    }
    void MinCircleCover(Point p[],int n){
        random_shuffle(p+1,p+1+n);
        Point c=p[1];
        double r=0;
        for (int i=2;i<=n;i++)
            if (sgn(Norm(c-p[i])-r)>0){
                c=p[i],r=0;
                for (int j=1;j<i;j++)
                    if (sgn(Norm(c-p[j])-r)>0){
                        c=(p[i]+p[j])/2.0;
                        r=Norm(c-p[i]);
                        for (int k=1;k<j;k++)
                            if (sgn(Norm(c-p[k])-r)>0){
                                c=GetCircleCenter(p[i],p[j],p[k]);
                                r=Norm(c-p[i]);
                            }
                    }
            }
        printf("%.10f\n%.10f %.10f",r,c.x,c.y);
    }
}
\end{lstlisting}





\newpage
\section{Others}
\subsection{sample}
\subsubsection{vimrc}
\begin{lstlisting}
set nocompatible
source $VIMRUNTIME/vimrc_example.vim
source $VIMRUNTIME/mswin.vim
nunmap <c-v>
set cindent
set number
set mouse=a
set tabstop=4
set shiftwidth=4
set cursorline
set guifont=Consolas:h14
inoremap kj <esc>
inoremap jk <esc>
inoremap { {}<left>
syntax enable
func! Compile()
    exec "w"
    exec "! g++ % -o %< -Wall -Wextra -Wshadow -Wconversion --std=c++14 -O2"
    exec "! ./%<"
endfunc
func! Debug()
    exec "w"
    exec "! g++- % -o %< -g -Wall --std=c++14 && gdb %<"
endfunc
func! AddTitle()
    call append(0,"// Cease to struggle and you cease to live")
    call append(1,"#include <bits/stdc++.h>")
    call append(2,"using namespace std;")
    call append(4,"int main() {")
    call append(5,"    ios::sync_with_stdio(0); cin.tie(0); cout.precision(6); cout << fixed;")
    call append(7,"    return 0;")
    call append(8,"}")
endfunc
map <F9> :call Compile()<CR>
map <F5> :call Debug()<CR>
map <F8> :call AddTitle()<CR>
\end{lstlisting}
\subsubsection{FastIO}
\begin{lstlisting}
namespace IO {
    const int MB = 1048576;
    const int RMAX = 16 * MB;
    const int WMAX = 16 * MB;
    #define getchar() *(rp++)
    #define putchar(x) (*(wp++) = (x))
    char rb[RMAX], *rp = rb,  wb[WMAX], *wp = wb;
    inline void init() {
        fread(rb, sizeof(char), RMAX, stdin);
    }
    template <class _T> inline void read(_T &_a) {
        _a = 0; register bool _f = 0; register int _c = getchar();
        while (_c < '0' || _c > '9') _f |= _c == '-', _c = getchar();
        while (_c >= '0' && _c <= '9') _a = _a * 10 + (_c ^ '0'), _c = getchar();
        _a = _f ? -_a : _a;
    }
    template <class _T> inline void write(_T _a) {
        static char buf[20], *top = buf;
        if (_a) {
            while (_a) {
                register _T tm = _a / 10;
                *(++top) = char(_a - tm * 10) | '0';
                _a = tm;
            }
            while (top != buf) putchar(*(top--));
        }
        else putchar('0');
    }
    void output() {
        fwrite(wb, sizeof(char), wp - wb, stdout);
    }
}
\end{lstlisting}
\subsection{Offline Algorithm}
\subsubsection{CDQ Divide and Conquer}
\begin{lstlisting}
struct Node {
    int x, y, z, ans;
    Node() {}
    Node(int _x, int _y, int _z):x(_x), y(_y), z(_z) {}
    bool operator < (const Node &b) const {
        if(y == b.y) {
            if(z == b.z) return x < b.x;
            return z < b.z;
        }
        return y < b.y;
    }
}A[MAXN], B[MAXN], C[MAXN];
int bit[MAXN];
void add(int k, int v) {
    for(; k <= m; k += k & -k) bit[k] = max(bit[k], v);
}
void clear(int k) {
    for(; k <= m; k += k & -k) bit[k] = 0;
}
int sum(int k) {
    int res = 0;
    for(; k; k -= k & -k) res = max(res, bit[k]);
    return res;
}
void solve(int l, int r) {
    if(l == r) {
        B[l] = A[l];
        return;
    }
    int mid = (l + r) >> 1;
    solve(l, mid);
    for(int i = mid + 1; i <= r; i++) B[i] = A[i];
    //sort(B + l, B + mid + 1);
    sort(B + mid + 1, B + r + 1);
    int L = l;
    for(int R = mid + 1; R <= r; R++) {
        while(L <= mid && B[L].y < B[R].y) add(B[L].z, B[L].ans), L++;
        A[B[R].x].ans = max(A[B[R].x].ans, sum(B[R].z - 1) + 1);
        B[R].ans = A[B[R].x].ans;
    }
    for(int i = l; i <= L; i++) clear(B[i].z);
    solve(mid + 1, r);
    L = l;
    int p = l, q = mid + 1;
    while(p <= mid || q <= r) {
        if(q > r || (p <= mid && B[p].y <= B[q].y)) C[L++] = B[p++];
        else C[L++] = B[q++];
    }
    for(int i = l; i <= r; i++) B[i] = C[i];
}
\end{lstlisting}
\subsubsection{Mo’s Algorithm}
\begin{lstlisting}
struct Node{
    int l, r, t, id;
    bool operator < (const Node& a) const {
        if(l /sz == a.l / sz) {
            if(r == a.r) return t < a.t;
            return r < a.r;
        }
        return l / sz < a.l / sz;
    }
}q[MAXN];
void solve() {
    while (t < q[i].t) addTime(t++, 1);
    while (t > q[i].t) addTime(--t, -1);
    while(L < q[i].l) add(L++, -1);
    while(L > q[i].l) add(--L, 1);
    while(R < q[i].r) add(++R, 1);
    while(R > q[i].r) add(R--, -1);
}
\end{lstlisting}
\subsubsection{Mo's Algorithm On Tree}
\begin{lstlisting}

\end{lstlisting}
\subsection{Randomized Algorithm}
\subsubsection{Simulated Annealing}
\begin{lstlisting}
void solve() {
    while(T > eps) {
        double alpha = ((rand() % 30001) / 15000.0) * pi;
        double theta = ((rand() % 10001) / 10000.0) * pi;
        tmp.x = cur.x + T * sin(theta) * cos(alpha);
        tmp.y = cur.y + T * sin(theta) * sin(alpha);
        tmp.z = cur.z + T * cos(theta);
        tmp.dis = cal(tmp);
        if(tmp.dis < cur.dis || (tmp.dis * 0.999 < cur.dis && (rand() & 7) == 7)) cur = tmp;
        //if(exp((cur.d - tmp.d)  / T) > ((double)rand() / RAND_MAX)) cur = tmp;

        T *= 0.999;
    }
}
\end{lstlisting}
}
\end{document}
